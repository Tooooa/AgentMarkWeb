\documentclass[tikz,border=6pt]{standalone}
\usepackage{ctex}
\usepackage{tikz}
\usetikzlibrary{arrows.meta,positioning,fit,backgrounds,calc}
%
\begin{document}
\begin{tikzpicture}[
  font=\sffamily\small,
  box/.style={draw=black, rounded corners=2pt, fill=white, align=center, minimum width=3.2cm, minimum height=0.95cm, inner sep=6pt},
  group/.style={draw=black, rounded corners=3pt, fill=none, inner sep=10pt},
  arrow/.style={-Latex, draw=black, line width=0.9pt},
  audit/.style={-Latex, draw=black, line width=0.8pt, densely dashed}
]
%
% top chain
\node[box] (u) at (0,0) {用户};
\node[box] (a) at (4.1,0) {业务前端\\或外部 Agent};
\node[box] (e) at (8.2,0) {Chrome 插件\\接入与采集};
\node[box] (p) at (12.3,0) {AgentMark Proxy};
\node[box] (l) at (16.4,0) {目标 LLM API};
%
\begin{pgfonlayer}{background}
  \node[group, fit=(u)(a)(e)(p)(l)] (g) {};
\end{pgfonlayer}
\node[anchor=south west, font=\bfseries] at (g.north west) {在线注入链路};
%
% request lane, left to right
\draw[arrow] (u) -- (a);
\draw[arrow] (a) -- (e);
\draw[arrow] (e) -- (p);
\draw[arrow] (p) -- node[above, font=\scriptsize] {转发} (l);
%
% response lane, right to left, slightly below
\draw[arrow] ([yshift=-6mm]l.west) -- ([yshift=-6mm]p.east);
\draw[arrow] ([yshift=-6mm]p.west) -- ([yshift=-6mm]e.east);
\draw[arrow] ([yshift=-6mm]e.west) -- ([yshift=-6mm]a.east);
%
% execution
\node[box] (x) at (4.1,-2.2) {执行 $\hat{b}_t$\\推进子目标};
\draw[arrow] (a.south) -- (x.north);
%
% core logic and audit sink
\node[box] (c) at (12.3,-2.2) {Core Logic\\采样与编码};
\draw[arrow] (p.south) -- (c.north);
\draw[arrow] (c.north) -- (p.south);
%
\node[box] (s) at (8.2,-2.2) {日志服务\\或数据库};
\draw[audit] (e.south) -- node[above, font=\scriptsize, fill=white, inner sep=1pt] {日志旁路} (s.north);
\draw[audit] (c.west) -- node[above, font=\scriptsize, fill=white, inner sep=1pt] {结构化事件} (s.east);
%
\end{tikzpicture}
\end{document}
